\documentclass{article}
\usepackage[utf8]{inputenc}
\usepackage[T1]{fontenc}
\usepackage{times}
\usepackage[margin=3cm]{geometry}

% German
%\usepackage[ngerman]{babel}
% English
\usepackage[english]{babel}

\usepackage[round,authoryear]{natbib}
\usepackage{amsmath,amssymb,amsthm}
\usepackage[hyphens]{url}
\usepackage{graphicx}
\usepackage{booktabs}



% German
%\newtheorem{definition}{Definition}
%\newtheorem{satz}{Satz}
% English
\newtheorem{definition}{Definition}
\newtheorem{theorem}{Theorem}

% TODO
\author{Alexander Lutsch\\Ephraim Siegfried\\Felix Andrist}
\title{ \Huge Cuisinventory }
\date{Fall Semester, 2023 \\ Computer Architecture}


\begin{document}
\maketitle

\section{Cuisinventory Introduction}
Cuisinventory is a grocery inventory managament system offering an easy way to keep track of consumption and additional product information.
It comes in the form of a station which has a barcode scanner and a weight scale with which you can interact. You can scan your groceries' barcode at the Cuisinventory station and
it will automatically fetch related product information like name, brands, allergens and conversation conditions. Additionally you will be able to weigh the groceries and with the fetched information about product quantity,
Cuisinventory will be able to provide percentage information about how much food is left for consumption. All of the inventory information is saved locally in a custom developed database, you can view it on a web application with which the station communicates.

\section{Materials and Libraries}

\section{User Manual}

\section{Implementation}

\section{Problems and Solutions}

\section{Summary}

\section{Sources}



\bibliographystyle{apalike}
\bibliography{references}
\end{document}
